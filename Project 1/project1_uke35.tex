\documentclass{article}

\usepackage[utf8]{inputenc}
\usepackage[T1]{fontenc}
\usepackage[english,norsk]{babel}
\usepackage{graphicx}
\usepackage{amsmath}
\usepackage{listings}

\setlength{\tabcolsep}{18pt}
\renewcommand{\arraystretch}{1.5}

\newcounter{excount}
\newenvironment{exercise}[1][]{\addtocounter{excount}{1} \noindent {\bf Exercise
\arabic{excount} \ \ #1}\hspace{2mm}}{\vspace{4mm}}

\begin{document}

\addtocounter{page}{0}

\title{Project 1 \\
      \large For the course FYS3150}
\date{\today \\
    \vspace{1mm}
    \large Week 35-37}

\author{Erik Grammeltvedt, Erlend Tiberg North and Alexandra Jahr Kolstad}

\maketitle

\begin{exercise}


\begin{itemize}
    \item[ \bf a)]

        In the exercise we are given the equation

        \begin{equation*}
            - \frac{v_{i+1} + v_{i-1} - 2 v_i}{h^2} = f_i \hspace{5mm} \textrm{for} ~  i = 1, 2, 3, ... , n
        \end{equation*}

        Rewrites the equation to
        \begin{align*}
            - (v_{i+1} + v_{i-1} - 2 v_i) &= h^2 f_i = \tilde{b}_i \\
            - v_{i+1} - v_{i-1} + 2 v_i &= \tilde{b}_i
        \end{align*}

        where in the exercise we are also given the correlation $\tilde{b}_i = h^2 f_i$, which is implemented here.

        Defines the equation for different values of the integer $i$ to get a set of equations. The exercise also gives the boundry conditions $v_0 = v_{n+1} = 0$.

        \begin{align*}
            i &= 1 : \hspace{5mm} - v_{1+1} - v_{1-1} + 2 v_1 = - v_2 - v_0 + 2 v_1 = - 0 + 2 v_1 - v_2 = \tilde{b}_1 \\
            i &= 2 : \hspace{5mm} - v_{2+1} - v_{2-1} + 2 v_2 = - v_3 - v_1 + 2 v_2 = - v_1 + 2 v_2 - v_3 = \tilde{b}_2 \\
            i &= 3 : \hspace{5mm} - v_{3+1} - v_{3-1} + 2 v_3 = - v_4 - v_2 + 2 v_3 = - v_2 + 2 v_3 - v_4 = \tilde{b}_3 \\
            \vdots \\
            i &= n : \hspace{5mm} - v_{n+1} - v_{n-1} + 2 v_n = - v_{n-1} + 2 v_n - 0 = \tilde{b}_n \\
        \end{align*}

        Equations can be rewritten as a matrix equation, which gives a matrix $A$ with integers as elements, a vector $\vec{v} = [v_1, v_2, v_3, ... , v_n]$ and another vector $\tilde{\vec{b}} = [\tilde{b}_1, \tilde{b}_2, \tilde{b}_3, ... , \tilde{b}_n]$.

        This gives the matrix equation

        \begin{equation*}
            A \vec{v} = \tilde{\vec{b}}
        \end{equation*}

        The matrix and the vectors are given as

        \begin{equation*}
            \begin{bmatrix}
                2 & -1 & 0 & 0 & \dots & 0 \\
                -1 & 2 & -1 & 0 & \dots & 0 \\
                0 & -1 & 2 & -1 & \dots & 0 \\
                \vdots & \vdots & \vdots & \vdots & \ddots & \vdots \\
                    0 & 0 & 0 & -1 & 2 & -1 \\
                    0 & 0 & 0 & 0 & -1 & 2 \\
            \end{bmatrix}
            \begin{bmatrix}
                v_1 \\
                v_2 \\
                v_3 \\
                \vdots \\
                v_{n-1} \\
                v_n \\
            \end{bmatrix}
            =
            \begin{bmatrix}
                \tilde{b}_1 \\
                \tilde{b}_2 \\
                \tilde{b}_3 \\
                \vdots \\
                \tilde{b}_{n-1} \\
                \tilde{b}_n \\
            \end{bmatrix}
        \end{equation*}

        Therefore the matrix equation has been proved.

%-----------slutten av a)--------------

\vspace{1cm}

    \item [\bf b)]

      i beregningene til erlend på ark er b diagonalelementene, g er b-vektoren og u er v-vektoren. i pythonkoden er d diagonalelementene, b er b-vektoren og v er v-vektoren. med tilde er algoritmediagonal




%---------slutten av b)-----------------

\vspace{1cm}

    \item [\bf c)]







%---------slutten av c)-----------------

\vspace{1cm}

    \item [\bf d)]








%---------slutten av d)-----------------

\vspace{1cm}

    \item [\bf e)]









%---------slutten av e)-----------------

\vspace{1cm}

\end{itemize}


%---------------------------

\end{exercise}


\end{document}
