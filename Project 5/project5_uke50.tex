\documentclass{article}

\usepackage[utf8]{inputenc}
\usepackage[T1]{fontenc}
\usepackage[norsk,english]{babel}   %Norsk først så engelsk, så engelsk blir prioritert
\usepackage{graphicx}
\usepackage{amsmath}        %For å kunne skrive matte
\usepackage{listings}       %For å kunne skrive inn kode med fin formatering
\usepackage{multicol}       %Importerer pakken for multikolonner til teksten
\usepackage[margin=2.54cm]{geometry}    %Definerer hva bredden til teksten er
\usepackage{wrapfig}    %Importerer pakken for å ha bildene i teksten
\usepackage[font = small]{caption}

%Definerer hyperlinker og dens farger
\usepackage{hyperref}
\hypersetup{
    colorlinks,
    citecolor=blue,
    filecolor=black,
    linkcolor=blue,
    urlcolor=blue
}


%-----------------------------------

%Definerer farger til kodeeksemplene i PDF-en
\usepackage{color}

\definecolor{codegreen}{rgb}{0,0.6,0}
\definecolor{codegray}{rgb}{0.5,0.5,0.5}
\definecolor{codepurple}{rgb}{0.58,0,0.82}
\definecolor{backcolour}{rgb}{0.95,0.95,0.92}

\lstdefinestyle{mystyle}{
    backgroundcolor=\color{backcolour},
    commentstyle=\color{codegreen},
    keywordstyle=\color{magenta},
    numberstyle=\tiny\color{codegray},
    stringstyle=\color{codepurple},
    basicstyle=\footnotesize,
    breakatwhitespace=false,
    breaklines=true,
    captionpos=b,
    keepspaces=true,
    numbers=left,
    numbersep=5pt,
    showspaces=false,
    showstringspaces=false,
    showtabs=false,
    tabsize=2
}

\lstset{style=mystyle}

%------------------------------------

\setlength{\parindent}{0pt} %Ingen indent automatisk for nye linjer
%\setlength{\columnsep}{2mm} %Column separation - til multicolumn

%\setlength{\arrayrulewidth}{1mm}   %Hvilken tykkelse tabellene skal ha
\setlength{\tabcolsep}{2mm}     %Lengden mellom hver kolonne
\renewcommand{\arraystretch}{1.5}   %Hvor stor avstand det skal være mellom radene

\iffalse    %midlertidig endre bredden på teksten
If you want to change this temporarily, you can write:
\savegeometry{mydefaultgeometry}
\newgeometry{margin=3in}
And then later you can call:
\loadgeometry{mydefaultgeometry}
\fi

%for å fjerne overskriften "refrences" som kommer automatisk når man bruker bibtex
\usepackage{etoolbox}
\patchcmd{\thebibliography}{\section*{\refname}}{}{}{}

% To avoid LaTeX re-positioning tables and images
\usepackage{float}
\restylefloat{table}

%----------------------------------------------------------------------------------------

\begin{document}

\addtocounter{page}{0}

\title{Project 5 \\
      \large For the course FYS3150}
\date{\today \\
    \vspace{1mm}
    \large Week XX - 51}

\author{Erik Grammeltvedt, Erlend Tiberg North and Alexandra Jahr Kolstad}

\maketitle

%\newpage

%------------Her starter skrivingen-----------------------------------------

%figurtekst under og tabelltekst over

%\begin{multicols}{2}


%-------------------- Abstract -------------------------------
\vspace{1cm}


\begin{center}

{\Large\textbf{Abstract}} \label{sec:Abstract}

\end{center}

An abstract where you give the main summary of your work


The abstract gives the reader a quick overview of what has been done and the most important results. Here is a typical example taken from a scientific article

We study the collective motion of a suspension of rodlike microswimmers in a two-dimensional film of viscoelastic fluids. We find that the fluid elasticity has a small effect on a suspension of pullers, while it significantly affects the pushers. The attraction and orientational ordering of the pushers are enhanced in viscoelastic fluids. The induced polymer stresses break down the large-scale flow structures and suppress velocity fluctuations. In addition, the energy spectra and induced mixing in the suspension of pushers are greatly modified by fluid elasticity.

\newpage

%------------------- Table of contents -----------------------

\vspace{1cm}

\tableofcontents

\vspace{1cm}

%-------------------- Introduction ------------------------------
\vspace{1cm}

\section{Introduction} \label{sec:Introduction}

An introduction where you explain the aims and rationale for the physics case and what you have done. At the end of the introduction you should give a brief summary of the structure of the report


What should I focus on? Introduction.
You don't need to answer all questions in a chronological order. When you write the introduction you could focus on the following aspects

Motivate the reader, the first part of the introduction gives always a motivation and tries to give the overarching ideas
What I have done
The structure of the report, how it is organized etc

%-------------------- Theory ------------------------------------
\vspace{1cm}

\section{Theory} \label{sec:Theory}



%--------------------- Method ------------------------------------
\vspace{1cm}

\section{Method} \label{sec:Method}

Theoretical models and technicalities. This is the methods section


What should I focus on? Methods sections.
Describe the methods and algorithms
You need to explain how you implemented the methods and also say something about the structure of your algorithm and present some parts of your code
You should plug in some calculations to demonstrate your code, such as selected runs used to validate and verify your results. The latter is extremely important!! A reader needs to understand that your code reproduces selected benchmarks and reproduces previous results, either numerical and/or well-known closed form expressions.

%--------------------- Results ----------------------------------
\vspace{1cm}

\section{Results} \label{sec:Results}

Results

What should I focus on? Results.
Present your results
An eventual reader should be able to reproduce your calculations if she/he wants to do so. All input variables should be properly explained.
Make sure that figures and tables should contain enough information in their captions, axis labels etc so that an eventual reader can gain a first impression of your work by studying figures and tables only.

\iffalse
  \begin{figure}[H]
      \centering
      \includegraphics[width = 11cm]{img/FILENAME}
      \caption{CAPTIONHERE}
      \label{fig:LABELHERE}
    \end{figure}

    \begin{table}[H]
      \centering
      \caption{CAPTION HERE}
      \vspace{2mm}
      \label{tab:LABELHERE}
      \begin{tabular}{|c|c|}
          \hline
           x & y\\
          \hline \hline
          x1 & y1 \\
          x2 & y2 \\
          x3 & y3 \\
          x4 & y4 \\
          x5 & y5 \\
          \hline
      \end{tabular} \\
      \hspace{0pt}\\
    \end{table}
\fi

%--------------- Discussion ---------------------------------------
\vspace{1cm}

\clearpage
\newpage

\section{Discussion} \label{sec:Discussion}

 and discussion


 Give a critical discussion of your work and place it in the correct context.
 Relate your work to other calculations/studies

%---------------Conclusion and perspective---------------------------
\vspace{1cm}

\section{Conclusion and perspective} \label{sec:Conclusion}

Conclusions and perspectives


What should I focus on? Conclusions.
State your main findings and interpretations
Try as far as possible to present perspectives for future work
Try to discuss the pros and cons of the methods and possible improvements

%----------------References----------------------------------------

\vspace{1cm}

\section{References} \label{sec:References}

\iffalse
What should I focus on? References.
Give always references to material you base your work on, either scientific articles/reports or books.
Refer to articles as: name(s) of author(s), journal, volume (boldfaced), page and year in parenthesis.
Refer to books as: name(s) of author(s), title of book, publisher, place and year, eventual page numbers
\fi

\begin{thebibliography}{}

\bibitem{task}
Morten H. Jensen (2019), \href{https://github.com/CompPhysics/ComputationalPhysics/blob/master/doc/Projects/2019/Project5/SolarSystem/pdf/SolarSystem.pdf}{Project 5}, Departement of Physics, University of Oslo, Norway

\bibitem{github}
Erik B. Grammeltvedt, Alexandra Jahr Kolstad, Erlend T. North (2019), \href{https://github.com/Erikbgram/Fys3150}{GitHub}, Students of Departement of Physics, University of Oslo, Norway

\bibitem{lecture_slides}
Morten H. Jensen (2015), \href{https://github.com/CompPhysics/ComputationalPhysics/blob/master/doc/Lectures/lectures2015.pdf}{Lecture slides for FYS3150}, Department of Physics, University of Oslo, Norway


\end{thebibliography}


%--------------Appendix---------------------------------------------
\vspace{1cm}


\appendix
\section{Appendix} \label{sec:Appendix}

Appendix with extra material


What should I focus on? additional material.
Additional calculations used to validate the codes
Selected calculations, these can be listed with few comments
Listing of the code if you feel this is necessary
You can consider moving parts of the material from the methods section to the appendix. You can also place additional material on your webpage.

\subsection{Mass-conversion} \label{app:mass}

\begin{align*}
    2 \cosh (x) &= e^x + e^{-x} \\
    2 \sinh (x) &= e^x - e^{-x}
\end{align*} \\

\begin{table}[H]
  \centering
  \caption{Astronomical data}
  \vspace{2mm}
  \label{tab:mass}
  \begin{tabular}{|c|c|c|c|}
      \hline
       Celestial Body & Mass in kg & Mass in $M_{\odot}$ & Distance to sun in AU\\
      \hline \hline \hline
      Sun     & $M_{\text{Sun}} = M_{\odot} = 2   \cdot10^{30}\;\text{kg}$ & 1 $M_{\odot}$ &  0    AU \\
      Mercury & $M_{\text{Mercury}}         = 3.3 \cdot10^{23}\;\text{kg}$ & 6060606.06 $M_{\odot}$ &  0.39 AU \\
      Venus   & $M_{\text{Venus}}           = 4.9 \cdot10^{24}\;\text{kg}$ & 408163.27 $M_{\odot}$ &  0.72 AU \\
      Earth   & $M_{\text{Earth}}           = 6   \cdot10^{24}\;\text{kg}$ & 333333.33 $M_{\odot}$ &  1    AU \\
      Mars    & $M_{\text{Mars}}            = 6.6 \cdot10^{23}\;\text{kg}$ & 3030303.03 $M_{\odot}$ &  1.52 AU \\
      Jupiter & $M_{\text{Jupiter}}         = 1.9 \cdot10^{27}\;\text{kg}$ & 1052.63 $M_{\odot}$ &  5.20 AU \\
      Saturn  & $M_{\text{Saturn}}          = 5.5 \cdot10^{26}\;\text{kg}$ & 3636.36 $M_{\odot}$ &  9.54 AU \\
      Uranus  & $M_{\text{Uranus}}          = 8.8 \cdot10^{25}\;\text{kg}$ & 22727.27 $M_{\odot}$ & 19.19 AU \\
      Neptune & $M_{\text{Neptune}}         = 1.03\cdot10^{26}\;\text{kg}$ & 19417.48 $M_{\odot}$ & 30.06 AU \\
      Pluto   & $M_{\text{Pluto}}           = 1.31\cdot10^{22}\;\text{kg}$ & 152671755.7 $M_{\odot}$ & 39.53 AU \\
      \hline
  \end{tabular} \\
  \hspace{0pt}\\
\end{table}


\iffalse
\begin{align*}
    2 \cosh (x) &= e^x + e^{-x} \\
    2 \sinh (x) &= e^x - e^{-x}
\end{align*} \\
\fi

%\clearpage


%----------------Slutten av dokumentet---------------------------------------


%\end{multicols}

\end{document}
