\documentclass{article}

\usepackage[utf8]{inputenc}
\usepackage[T1]{fontenc}
\usepackage[norsk,english]{babel}   %norsk først så engelsk, så engelsk blir prioritert
\usepackage{graphicx}
\usepackage{amsmath}
\usepackage{listings}

%Definerer hyperlinker og dens farger
\usepackage{hyperref}
\hypersetup{
    colorlinks,
    citecolor=black,
    filecolor=black,
    linkcolor=blue,
    urlcolor=blue
}

%-----------------------------------

%Definerer farger til kodeeksemplene i PDF-en
\usepackage{color}

\definecolor{codegreen}{rgb}{0,0.6,0}
\definecolor{codegray}{rgb}{0.5,0.5,0.5}
\definecolor{codepurple}{rgb}{0.58,0,0.82}
\definecolor{backcolour}{rgb}{0.95,0.95,0.92}

\lstdefinestyle{mystyle}{
    backgroundcolor=\color{backcolour},
    commentstyle=\color{codegreen},
    keywordstyle=\color{magenta},
    numberstyle=\tiny\color{codegray},
    stringstyle=\color{codepurple},
    basicstyle=\footnotesize,
    breakatwhitespace=false,
    breaklines=true,
    captionpos=b,
    keepspaces=true,
    numbers=left,
    numbersep=5pt,
    showspaces=false,
    showstringspaces=false,
    showtabs=false,
    tabsize=2
}

\lstset{style=mystyle}

%------------------------------------

\setlength{\parindent}{0pt}

\setlength{\tabcolsep}{18pt}
\renewcommand{\arraystretch}{1.5}

\begin{document}

\addtocounter{page}{0}

\title{Project 1 \\
      \large For the course FYS3150}
\date{\today \\
    \vspace{1mm}
    \large Week 35 - 37}

\author{Erik Grammeltvedt, Erlend Tiberg North and Alexandra Jahr Kolstad}

\maketitle

%------------Her starter skrivingen-----------------------------------------
\vspace{1cm}

\tableofcontents

\vspace{1cm}

%----------Abstract-------------------------------


\section{Abstract} \label{sec:Abstract}




%--------------Introduction------------------------------
\vspace{1cm}

\section{Introduction} \label{sec:Introduction}


All programs are found at our \href{https://github.com/Erikbgram/Fys3150}{GitHub-repository}. \\


%---------------Method------------------------------------
\vspace{1cm}

\section{Method} \label{sec:Method}



\subsection{Exercise a)} \label{sec:Method a)}




\subsection{Exercise b)} \label{sec:Method b)}

  \subsubsection{Calculations}

    Det under som ikke er mulig å lese blir kommentert ut:

    \iffalse

    dovkdfv
    fvokdfv
    odkv
    dfvkd
    ofvkdfovkdf

    \fi

    Ferdig kommentert ut. 


%    \begin{equation*} \label{eq:fullmatrixeqbackward}
%      \begin{bmatrix}
%        d_1 & c_1 & 0 & 0 & \dots & 0 \\
%        0 & \tilde{d}_2 & c_2 & 0 & \dots & 0 \\
%        0 & 0 & \tilde{d}_3 & c_3 & \dots & 0 \\
%        \vdots & \vdots & \vdots & \vdots & \ddots & \vdots \\
%        0 & 0 & 0 & 0 & \tilde{d}_{n-1} & c_{n-1} \\
%        0 & 0 & 0 & 0 & 0 & \tilde{d}_n \\
%      \end{bmatrix}
%      \begin{bmatrix}
%        v_1 \\
%        v_2 \\
%        v_3 \\
%        \vdots \\
%        v_{n-1} \\
%        v_n \\
%      \end{bmatrix}
%      =
%      \begin{bmatrix}
%        \tilde{b}_1 \\
%        \tilde{b}_2 \\
%        \tilde{b}_3 \\
%        \vdots \\
%        \tilde{b}_{n-1} \\
%        \tilde{b}_n \\
%      \end{bmatrix}
%    \end{equation*} \\


  \subsubsection{The programming}




\subsection{Exercise c)} \label{sec:Method c)}

  \subsubsection{Calculations}






  \subsubsection{The programming}




\subsection{Exercise d)} \label{sec:Method d)}

  \subsubsection{Calculations}




  \subsubsection{The programming}



\subsection{Exercise e)} \label{sec:Method e)}


  \subsubsection{Calculations}




  \subsubsection{The programming}





%--------------Results and discussion------------------------------
\vspace{1cm}

\section{Results and discussion} \label{sec:Results}

  Our results are as shown in the \nameref{sec:Appendix}. We also have \texttt{.txt}-files for all the raw data generated by the projects up on \href{https://github.com/Erikbgram/Fys3150}{GitHub}. \\

  \subsection{Exercise a)} \label{sec:Results a)}



  \subsection{Exercise b)} \label{sec:Results b)}



  \subsection{Exercise c)} \label{sec:Results c)}




%  \begin{lstlisting}[language=C++]
%    std::cout << "index, value\n";
%  for(int i = 0; i < n; i++){
%      std::cout << "    " << i << ", " << d_new[i] << std::endl;
%  }
%  \end{lstlisting}



%  \begin{verbatim}
%index, value
%    0, 2
%    1, 1.5
%    2, 1.33333
%    3, 1.25
%    4, 1.2
%    5, 1.16667
%    6, 1.14286
%    7, 1.125
%    8, 1.11111
%    9, 1.1
%index, value
%    0, 2
%    1, 2
%    2, 1.5
%    3, 1.33333
%    4, 1.25
%    5, 1.2
%    6, 1.16667
%    7, 1.14286
%    8, 1.125
%    9, 1.11111
%  \end{verbatim}


  \subsection{Exercise d)} \label{sec:Results d)}




  \subsection{Exercise e)} \label{sec:Results e)}





%  \begin{table}[ht] \label{tab:exec_time}
%    \centering
%      \caption{Execution time for the two methods.}
%      \vspace{2mm}
%      \begin{tabular}{|c|c|c|}
%        \hline
%        $n$    &   Tridiagonal      &  LU-Decomposition  \\
%        \hline \hline
%        10   & $2.00\cdot10^{-7}$ & $3.17\cdot10^{-4}$ \\
%        100  & $1,40\cdot10^{-6}$ & $1.40\cdot10^{-3}$ \\
%        1000 & $1.44\cdot10^{-5}$ & $3.36\cdot10^{-2}$ \\
%        \hline
%      \end{tabular} \\
%      \hspace{0pt}\\
%  \end{table}



%---------------Conclusion and perspective---------------------------
\vspace{1cm}

\section{Conclusion and perspective} \label{sec:Conclusion}



%--------------Appendix---------------------------------------------
\vspace{1cm}

\section{Appendix} \label{sec:Appendix}


%\begin{figure}[ht]
%	\centering
%	\includegraphics[width = 11cm]{program/data10.png}
%	\caption{The plot of the different algorithms for $n = 10$. }
%  \label{fig:data10png}
%\end{figure}



\clearpage

%----------------Refrences----------------------------------------
\vspace{1cm}

\section{References} \label{sec:References}


\href{https://github.com/Erikbgram/Fys3150}{Our GitHub-repository}. \\

\href{https://github.com/CompPhysics/ComputationalPhysics/blob/master/doc/Lectures/lectures2015.pdf}{Link to lecture slides in FYS3150 - Computational Physics}. See page 168 and the rest of chapter \textbf{6.4 Linear Systems} for theory behind the tridiagonal matrix algorithm.






%----------------Slutten av dokumentet---------------------------------------



\end{document}
