\documentclass{article}
\usepackage[utf8]{inputenc}

\title{Project 3}
\author{erikbgram }
\date{October 2019}

\begin{document}

\maketitle

\section{Introduction}
\section{Theory}

\subsection{Theory of Gaussian Quadrature, the Gauss-Legendre- Gauss-Laguerre- method}

The Gauss-Legendre method is based on the more general Gaussian Quadrature method witch uses Taylor series to solve an integral. The main idea is to generate weights, by solving sets of linear equations. For N pints in the Taylor series we get N weights. This weights are used in a weight function in order to approximate the integral.The theory behind Gaussian Quadrature is to obtain the weights by using orthogonal polynomials. This polynomials are orthogonal at certain intervals. For instance from [3,7] as an example. we can use these orthogonal polynomials in order to insure a smooth integral for are graph. The $x_i$ values are chosen arbitrary within the given interval. Together with the weights this gives us 2N parameters that can be used to solve the integral.      
\newline
\newline

$I = \int_{a}^{b} f(x) = \int_{a}^{b}W(x)g(x)dx \approx  \sum_{n=1}^{N}  \omega_i g(x_i) $ (2) 
\newline
\newline

In equation (1) we have the weight function W(x) and g(x) is a orthogonal polynomial that gives a smooths graph. Then we have the sum the weights, $\omega_i$, and the orthogonal polynomial $g(x_i)$ with a number of $x_i$ values within the given interval. 
\newline
\newline

In order to go from Gaussian Quadrature to the Gauss-Legendre method a unit change is needed. Because the Gauss-Legendre method uses only the integral from [-1,1] and later apply the Gaussian Quadrature. A unit change is needed in order to use the Gauss-Legendre method.  
\newline
\newline

Want to change t too $t = \frac{b-a}{2}x + \frac{a+b}{2}$
\newline
\newline

$\int_{a}^{b} f(t) dt = \frac{b-a}{2}\int_{-1}^{1} f(\frac{b-a}{2}x + \frac{a+b}{2}) dx$ (3)
\newline
\newline

Now exchanging this new integral, (2), into the Gaussian Quadrature, (1), the following estimate of the integral becomes,(3). 
\newline
\newline

$\int_{a}^{b} f(t) dt \approx \frac{b-a}{2} \sum_{n=1}^{N} \omega_i f(\frac{b-a}{2}x + \frac{a+b}{2}) dx$ (4)
\newline
\newline

The Guass-Leguerra method is based on the Gaussian Quadrature method, but it is mainly maid for integrals of type (5).    
\newline
\newline
 $\int_{0}^{\infty} e^{-x} f(x)dx$ (5)
\newline
\newline
Incorporating equation (5) into the Gaussian Quadrature, (2), gilds a result of weights with a specific calculation method.
\newline
\newline
$\int_{0}^{\infty} e^{-x} f(x) dx \approx \sum_{n=1}^{N} \omega_if(x_i) $
\newline
\newline
The weights are calculated by the folowing.
\newline
\newline
$\omega_i = \frac{x_i}{(n+1)² [L_{n+1}(x_i)]²}$ (6)
\newline
\newline
Where l $L_n(x)$ is the Leguerra polynomial. according to WolframMathworld the Leguerra polynomials "are solutions $L_n(x)$ to the Laguerre differential equation with $\gamma=0"$ (REF TIL LINK!!!!!). These polynomials are given as:
\newline
\newline
$L_n(x) = \frac{e^x}{n!}\frac{d^n}{dx^n}(x^n e^{-x}$
\newline
\newline
Even though this method is manly used on functions containing $e^(-x)$. It can also be used on any function by using algebraic manipulation.
\newline
\newline
$\int_{0}^{\infty} f(x) = \int_{0}^{\infty} f(x) e^{x}e^{-x}$ 
\newline
\newline
set g(x) = $f(x) e^x$ and we are left with (7)
\newline
\newline
$\int_{0}^{\infty} g(x) e^{-x}$  (7)
\newline
\newline
This algebraic trick allows the Gauss-Laguerre method to calulate the integral of any given function.   

\subsection{Theory of the Monte Carlo Method}

The Monte Carlo integration method is based on summing the value of many different points on the graph together. After summing them together they are divided by the number of points used to get the precise integral. The points, $x_i$ are chosen at random within the given interval.   
\newline
\newline
$\int_{a}^{b} f(x) = \frac{1}{N} \sum_{n=1}^{N} \omega_if(x) $ (5)
\newline
\newline
In (5) the weight, $w_i$, is different depending on what method is preferred. For instance one can apply the Simpsons method here. As an example the brute force Monte Carlo method uses $w_i = 1$. Then we get equation (6).
\newline
\newline
$\int_{a}^{b} f(x) = \frac{1}{N} \sum_{n=1}^{N} f(x_{i-1/2}) $ (6)
\newline
\newline
How ever the brute force Monte Carlo is not necessarily the most efficient method. Depending on the problem, other methods can be applied as mentioned above.












 
\end{document}
