\documentclass{article}
\usepackage[utf8]{inputenc}

\title{Project 3}
\author{erikbgram }
\date{October 2019}

\begin{document}

\maketitle

\section{Introduction}
\section{Theory}

\subsection{Theory of Gaussian Quadrature and the Gauss-Legendre- method}

The Gauss-Legendre method is based on the more general Gaussian Quadrature method witch uses Taylor series to solve an integral. The main idea is to generate weights, by solving sets of linear equations. For N pints in the Taylor series we get N weights. This weights are used in a weight function in order to approximate the integral.The theory behind Gaussian Quadrature is to obtain the weights by using orthogonal polynomials. This polynomials are orthogonal at certain intervals. For instance from [3,7] as an example. we can use these orthogonal polynomials in order to insure a smooth integral for are graph. The $x_i$ values are chosen arbitrary within the given interval. Together with the weights this gives us 2N parameters that can be used to solve the integral.      
\newline
\newline

$I = \int_{a}^{b} f(x) = \int_{a}^{b}W(x)g(x)dx \approx  \sum_{n=1}^{N}  \omega_i g(x_i) $ (1) 
\newline
\newline

In equation (1) we have the weight function W(x) and g(x) is a orthogonal polynomial that gives a smooths graph. Then we have the sum the weights, $\omega_i$, and the orthogonal polynomial $g(x_i)$ with a number of $x_i$ values within the given interval. 
\newline
\newline

In order to go from Gaussian Quadrature to the Gauss-Legendre method a unit change is needed. Because the Gauss-Legendre method uses only the integral from [-1,1] and later apply the Gaussian Quadrature. A unit change is needed in order to use the Gauss-Legendre method.  
\newline
\newline

Want to change t too $t = \frac{b-a}{2}x + \frac{a+b}{2}$
\newline
\newline

$\int_{a}^{b} f(t) dt = \frac{b-a}{2}\int_{-1}^{1} f(\frac{b-a}{2}x + \frac{a+b}{2}) dx$ (2)
\newline
\newline

Now exchanging this new integral, (2), into the Gaussian Quadrature, (1), the following estimate of the integral becomes,(3). 
\newline
\newline

$\int_{a}^{b} f(t) dt \approx \frac{b-a}{2} \sum_{n=1}^{N} \omega_i f(\frac{b-a}{2}x + \frac{a+b}{2}) dx$ (3)


\subsection{Theory of the Montecarlo Method}
