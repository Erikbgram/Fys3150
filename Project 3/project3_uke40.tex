\documentclass{article}

\usepackage[utf8]{inputenc}
\usepackage[T1]{fontenc}
\usepackage[norsk,english]{babel}   %norsk først så engelsk, så engelsk blir prioritert
\usepackage{graphicx}
\usepackage{amsmath}
\usepackage{listings}
\usepackage{multicol}
\usepackage[margin=2.54cm]{geometry}
\usepackage{wrapfig}

%Definerer hyperlinker og dens farger
\usepackage{hyperref}
\hypersetup{
    colorlinks,
    citecolor=black,
    filecolor=black,
    linkcolor=blue,
    urlcolor=blue
}

%-----------------------------------

%Definerer farger til kodeeksemplene i PDF-en
\usepackage{color}

\definecolor{codegreen}{rgb}{0,0.6,0}
\definecolor{codegray}{rgb}{0.5,0.5,0.5}
\definecolor{codepurple}{rgb}{0.58,0,0.82}
\definecolor{backcolour}{rgb}{0.95,0.95,0.92}

\lstdefinestyle{mystyle}{
    backgroundcolor=\color{backcolour},
    commentstyle=\color{codegreen},
    keywordstyle=\color{magenta},
    numberstyle=\tiny\color{codegray},
    stringstyle=\color{codepurple},
    basicstyle=\footnotesize,
    breakatwhitespace=false,
    breaklines=true,
    captionpos=b,
    keepspaces=true,
    numbers=left,
    numbersep=5pt,
    showspaces=false,
    showstringspaces=false,
    showtabs=false,
    tabsize=2
}

\lstset{style=mystyle}

%------------------------------------

\setlength{\parindent}{0pt}
\setlength{\columnsep}{5mm} %column separation
\setlength{\columnsep}{10mm}

\setlength{\tabcolsep}{18pt}
\renewcommand{\arraystretch}{1.5}

\iffalse
If you want to change this temporarily, you can write:
\savegeometry{mydefaultgeometry}
\newgeometry{margin=3in}
And then later you can call:
\loadgeometry{mydefaultgeometry}
\fi

\begin{document}

\addtocounter{page}{0}

\title{Project 3 \\
      \large For the course FYS3150}
\date{\today \\
    \vspace{1mm}
    \large Week 40 - ??}

\author{Erik Grammeltvedt, Erlend Tiberg North and Alexandra Jahr Kolstad}

\maketitle

%\newpage

%------------Her starter skrivingen-----------------------------------------

%\begin{multicols}{2}

\newpage
\clearpage

\textbf{Kommentarer fra project 1 på devilry:}

\begin{itemize}

\item Abstract: short motivation and presentation of the results and the findings \\

\item Introduction: you want to motive the reader about the problem and why you want solve it \\

\item Theory: explaining the theory behind the solution method and the problem \\

\item Method/implementation: how you implement the solution in order to fix/solve the problem \\

\item Results/graphs/tables: presenting the results \\

\item Discussion: Discussing the result from previous section \\

\item Conclusion: concluding the findings, your neutral opinion, etc… and future work \\

\item Appendix: How you derived your method, theory, etc… , altså utledning av ting i teori som ikke spesifikt er et bevis \\

\end{itemize}


Ting å gjøre for de ulike oppgavene:
\begin{itemize}

  \item 3a: beregne integralet, how many mesh points, lage et plott for å sjekke om grensene er passende å bruke \\

  \item 3b: finne grensene, erstatte Gauss-Legendre metoden med Laguerre polynomer, sammenligne med resultater fra a \\

  \item 3c: nå bruke brute force Monte Carlo, sammenligne resultatene med tidligere \\

  \item 3d: forbedre Monte Carlo med bruk av importance sampling, kommentere resultatene, lage en liste over tidene, sammenligne resultatene \\

  \item 3e: parallellisere koden fra 3d med openMPI eller MPI, kommenter resultatene (hovedsakelig i tiden brukt)

\end{itemize}


%-------------------- Abstract -------------------------------
\vspace{1cm}


\begin{center}

{\Large\textbf{Abstract}} \label{sec:Abstract}

\end{center}

\vspace{5mm}


%\begin{abstract}

hensikt: tilnærme løsningen til integralet så best som mulig 5 pi**2 / 16**2 . \\

In this project we will compute an integral with different numerical methods of integration, while comparing these values to the exact answer. Our integral is

\begin{equation}
    \left\langle \frac{1}{| \vec{r_1} - \vec{r_2} |} \right\rangle = \int_{-\infty} ^\infty d \vec{r_1} d \vec{r_2} \hspace{1mm} e^{- 2 \alpha (r_1 + r_2)} \hspace{1mm} \frac{1}{| \vec{r_1} - \vec{r_2} |}
\end{equation}

which is not normalized. The answer to the integral is $\frac{5 \pi^2}{16^2} \approx 0.192765710958777$. The numerical integration methods we will use in this project are Gauss-Legendre quadrature, Gauss-Laguerre quadrature, brute force Monte Carlo, Monte Carlo with importance sampling, and Monte Carlo with parallization. \\

In this numerical study we are going to use different numerical integration methods to approximate the ground state correlation energy between two electrons in a helium atom. The main interest and the goal of this study is to look at how the methods compare with different amount of mesh points and integration limits. The methods in question are Gauss-Legendre Quadrature, Gauss-Laguerre Quadrature, and some variations of the Monte Carlo Method.
The study was a success and proved Monte Carlo to be the fastest and most accurate method. Being based on big data, it required many more sampling points. However, as it did not manually calculate the integral, it was much more efficient.\\
SETT INN NOEN TALL HER

\begin{verbatim}

1x-193-157-253-38:project3 Alexandra$ ./ni.o 1000000000 3

Gauss-Legendre quad =     0.00000000000000
Exact answer =          0.192765710958777
Error =                 0.192765710958777
Time used by Gauss-Legendre = 1.73000000000000E-07s

Gauss-Laguerre quad =     0.00000000000000
Exact answer =           0.192765710958777
Error =                   0.192765710958777
Time used by Gauss-Laguerre = 1.17000000000000E-07s

Brute Force Monte Carlo =    0.193462755727606
Exact answer =           0.192765710958777
Error =                0.000697044768829452
Time used by Brute Force Monte Carlo = 653.636127568000s

Spherical Monte Carlo w/ Imp.Sampling =    0.192692246996587
Exact answer =           0.192765710958777
Error =                7.34639621890743E-05
Time used by Spherical Monte Carlo w/ Imp.Sampling = 772.072297750000s

Standard deviation BMC = 0.00111909179849692
Standard deviation SMC = 3.31038411626575E-05

---------------------------------------------------

1x-193-157-253-38:program Alexandra$ mpirun -n 2 ./ni2.o 1000000 3

Parallelized Spherical Monte Caro w/ Imp. Sampling = 0.192536
Parallelized Spherical Monte Caro w/ Imp. Sampling std = 0.000295753
Time = 0.405455 on number of processors: 2

Gauss-Legendre quad =     0.00000000000000
Exact answer =          0.192765710958777
Error =                 0.192765710958777
Time used by Gauss-Legendre = 1.35000000000000E-07 s

Gauss-Laguerre quad =     0.00000000000000
Exact answer =           0.192765710958777
Error =                   0.192765710958777
Time used by Gauss-Laguerre = 7.00000000000000E-08 s

Brute Force Monte Carlo =    0.199887417052828
Exact answer =           0.192765710958777
Error =                 0.00712170609405155
Time used by Brute Force Monte Carlo = 0.650475530000000 s

Spherical Monte Carlo w/ Imp.Sampling =    0.193504891519357
Exact answer =           0.192765710958777
Error =                0.000739180560580976
Time used by Spherical Monte Carlo w/ Imp.Sampling = 0.871346541000000 s

Standard deviation BMC = 0.000149923598473426
Standard deviation SMC = 0.000199594116160946


Gauss-Legendre quad =     0.00000000000000
Exact answer =          0.192765710958777
Error =                 0.192765710958777
Time used by Gauss-Legendre = 1.53000000000000E-07 s

Gauss-Laguerre quad =     0.00000000000000
Exact answer =           0.192765710958777
Error =                   0.192765710958777
Time used by Gauss-Laguerre = 7.50000000000000E-08 s

Brute Force Monte Carlo =    0.179064317908673
Exact answer =           0.192765710958777
Error =                  0.0137013930501036
Time used by Brute Force Monte Carlo = 0.661376967000000 s

Spherical Monte Carlo w/ Imp.Sampling =    0.194483818491421
Exact answer =           0.192765710958777
Error =                 0.00171810753264470
Time used by Spherical Monte Carlo w/ Imp.Sampling = 0.876190826000000 s

Standard deviation BMC = 0.000116998041926188
Standard deviation SMC = 0.000216150932593955


\end{verbatim}

All programs are found at our \href{https://github.com/Erikbgram/Fys3150}{GitHub-repository}. \\

%\end{abstract}

%\end{center}

\newpage

%------------------- Table of contents -----------------------

\vspace{1cm}

\tableofcontents

\vspace{1cm}

%-------------------- Introduction ------------------------------
\vspace{1cm}

\section{Introduction} \label{sec:Introduction}

The integral we are evaluating comes from the solution of schrödinger's equation for a simplified case of determining the ground state correlation energy between two electrons in a helium atom.

\begin{equation}
    \left\langle \frac{1}{| \vec{r_1} - \vec{r_2} |} \right\rangle = \int_{-\infty} ^\infty d \vec{r_1} d \vec{r_2} \hspace{1mm} e^{- 2 \alpha (r_1 + r_2)} \hspace{1mm} \frac{1}{| \vec{r_1} - \vec{r_2} |}
\end{equation}

The integral is not properly normalized. However, that is not important for the study.
If you are interested in how the integral is found, see \cite{task}.\\
Our aim is, as written in abstract, to approximate this integral using two types of Gaussian Quadrature, and some variations of Monte Carlo. We have created a code that evaluates the integral using each method and compares them. Alongside the actual value we get the absolute value, and time used. Using this we present our data showing that Monte Carlo massively outperforms the other methods.\\
The report will go through the theory and methods behind our study and following that we will present our results and discuss them.

%-------------------- Theory ------------------------------------
\vspace{1cm}

\section{Theory} \label{sec:Theory}


  \begin{equation*} \label{eq:fullmatrixeq}
    \begin{bmatrix}
        d & a & 0 & \dots & 0 & 0 \\
        a & d & a & \dots & 0 & 0 \\
        0 & a & d & \dots & 0 & 0 \\
        \vdots & \vdots & \vdots & \ddots & \vdots & \vdots \\
        0 & 0 & 0 & a & d & a \\
        0 & 0 & 0 & 0 & a & d \\
    \end{bmatrix}
    \begin{bmatrix}
        u_1 \\
        u_2 \\
        u_3 \\
        \vdots \\
        u_{N-2} \\
        u_{N-1} \\
    \end{bmatrix}
      = \lambda
    \begin{bmatrix}
        u_1 \\
        u_2 \\
        u_3 \\
        \vdots \\
        u_{N-2} \\
        u_{N-1} \\
    \end{bmatrix}
  \end{equation*} \\

HER SKAL ERIK SITT INN

%--------------------- Method ------------------------------------
\vspace{1cm}

\section{Method} \label{sec:Method}

The methods used in this project are Gauss-Legendre quadrature, Gauss-Laguerre quadrature, brute force Monte Carlo, Monte Carlo with importance sampling, and Monte Carlo with parallization.

\subsection{Gauss-Legendre}

This method is excepted to give to worst approximation to the integral. This is because it uses approximations and other short cuts to more easily compute the integral. For instance the limits are originally given by $\pm \infty$ and in this method they have to be approximated to finite numbers.

integrand diverges.

THIS SHOULD ALSO BE ERIK SITT

%--------------------- Results ----------------------------------
\vspace{1cm}

\section{Results} \label{sec:Results}

  \textit{Our results are as shown in the \nameref{sec:Appendix}}. We also have \texttt{.txt}-files for all the raw data generated by the projects up on \href{https://github.com/Erikbgram/Fys3150}{GitHub}. \\

\begin{itemize}

  \item How many mesh points do you need before the results converges at the level of the third leading digit?

\end{itemize}

  Burde ha at lambda er 2 siden det gir best resultater, se plottet fra plot\_data.txt.


\iffalse

  \begin{figure}[ht]
  	\centering
    \includegraphics[width = 11cm]{iterations-stats.png}
    %\includegraphics[width = \linewidth]{iterations-stats.png}
  	\caption{The plot of iterations for the Jacobi method as function of the dimension $n$ of the matrix \textbf{A}. }
    \label{fig:iterationspng}
  \end{figure}

  \begin{figure}[ht]
    \centering
    \includegraphics[width = 11cm]{timespan-stats.png}
    %\includegraphics[width = \linewidth]{timespan-stats.png}
    \caption{The plot of the time the function \texttt{eig\_sym} from Armadillo uses and the time Jacobi method uses as functions of the dimension $n$ of the matrix \textbf{A}. }
    \label{fig:timespanpng}
  \end{figure}

\fi



%\vspace{3cm}



  \begin{table}[ht] \label{tab:exec_time}
    \centering
      \caption{Execution time for the two methods.}
      \vspace{2mm}
      \begin{tabular}{|c|c|c|}
        \hline
        $n$    &   Tridiagonal      &  LU-Decomposition  \\
        \hline \hline
        10   & verdier & $3.17\cdot10^{-4}$ \\
        100  & $1,40\cdot10^{-6}$ & $1.40\cdot10^{-3}$ \\
        1000 & $1.44\cdot10^{-5}$ & $3.36\cdot10^{-2}$ \\
        \hline
      \end{tabular} \\
      \hspace{0pt}\\
  \end{table}


%--------------- Discussion ---------------------------------------
\vspace{1cm}

\section{Discussion} \label{sec:Discussion}




%---------------Conclusion and perspective---------------------------
\vspace{1cm}

\section{Conclusion and perspective} \label{sec:Conclusion}



%--------------Appendix---------------------------------------------
\vspace{1cm}

\section{Appendix} \label{sec:Appendix}






%\clearpage

%----------------References----------------------------------------
\vspace{1cm}

\section{References} \label{sec:References}

\href{https://github.com/CompPhysics/ComputationalPhysics/blob/master/doc/Projects/2019/Project2/pdf/Project2.pdf}{Link to the PDF for Project 2}. \\

\href{https://github.com/Erikbgram/Fys3150}{Our GitHub-repository}. \\

\href{https://github.com/CompPhysics/ComputationalPhysics/blob/master/doc/Lectures/lectures2015.pdf}{Link to lecture slides in FYS3150 - Computational Physics}. \\

\href{http://arma.sourceforge.net/docs.html#eig_sym}{Offical Armadillo website for documentation of all contents in the library}. \\

\href{https://journals.aps.org/pra/pdf/10.1103/PhysRevA.48.3561}{Analytical results for specific oscillator frequencies}. \\

\begin{thebibliography}{}
\bibitem{task}
Morten H. Jensen (2019) \emph{\href{https://github.com/CompPhysics/ComputationalPhysics/blob/master/doc/Projects/2019/Project3/pdf/Project3.pdf}{Project 3}}, Departement of Physics, University of Oslo, Norway

\bibitem{github}
Erik B. Grammeltvedt, Alexandra Jahr Kolstad, Erlend T. North (2019) \emph{\href{https://github.com/Erikbgram/Fys3150}{GitHub}}, Students of Departement of Physics, University of Oslo, Norway
\end{thebibliography}




%----------------Slutten av dokumentet---------------------------------------



%\end{multicols}

\end{document}
